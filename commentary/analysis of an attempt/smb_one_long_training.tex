% Options for packages loaded elsewhere
\PassOptionsToPackage{unicode}{hyperref}
\PassOptionsToPackage{hyphens}{url}
\documentclass[
  landscape]{article}
\usepackage{xcolor}
\usepackage[margin=1in]{geometry}
\usepackage{amsmath,amssymb}
\setcounter{secnumdepth}{-\maxdimen} % remove section numbering
\usepackage{iftex}
\ifPDFTeX
  \usepackage[T1]{fontenc}
  \usepackage[utf8]{inputenc}
  \usepackage{textcomp} % provide euro and other symbols
\else % if luatex or xetex
  \usepackage{unicode-math} % this also loads fontspec
  \defaultfontfeatures{Scale=MatchLowercase}
  \defaultfontfeatures[\rmfamily]{Ligatures=TeX,Scale=1}
\fi
\usepackage{lmodern}
\ifPDFTeX\else
  % xetex/luatex font selection
\fi
% Use upquote if available, for straight quotes in verbatim environments
\IfFileExists{upquote.sty}{\usepackage{upquote}}{}
\IfFileExists{microtype.sty}{% use microtype if available
  \usepackage[]{microtype}
  \UseMicrotypeSet[protrusion]{basicmath} % disable protrusion for tt fonts
}{}
\makeatletter
\@ifundefined{KOMAClassName}{% if non-KOMA class
  \IfFileExists{parskip.sty}{%
    \usepackage{parskip}
  }{% else
    \setlength{\parindent}{0pt}
    \setlength{\parskip}{6pt plus 2pt minus 1pt}}
}{% if KOMA class
  \KOMAoptions{parskip=half}}
\makeatother
\usepackage{color}
\usepackage{fancyvrb}
\newcommand{\VerbBar}{|}
\newcommand{\VERB}{\Verb[commandchars=\\\{\}]}
\DefineVerbatimEnvironment{Highlighting}{Verbatim}{commandchars=\\\{\}}
% Add ',fontsize=\small' for more characters per line
\usepackage{framed}
\definecolor{shadecolor}{RGB}{248,248,248}
\newenvironment{Shaded}{\begin{snugshade}}{\end{snugshade}}
\newcommand{\AlertTok}[1]{\textcolor[rgb]{0.94,0.16,0.16}{#1}}
\newcommand{\AnnotationTok}[1]{\textcolor[rgb]{0.56,0.35,0.01}{\textbf{\textit{#1}}}}
\newcommand{\AttributeTok}[1]{\textcolor[rgb]{0.13,0.29,0.53}{#1}}
\newcommand{\BaseNTok}[1]{\textcolor[rgb]{0.00,0.00,0.81}{#1}}
\newcommand{\BuiltInTok}[1]{#1}
\newcommand{\CharTok}[1]{\textcolor[rgb]{0.31,0.60,0.02}{#1}}
\newcommand{\CommentTok}[1]{\textcolor[rgb]{0.56,0.35,0.01}{\textit{#1}}}
\newcommand{\CommentVarTok}[1]{\textcolor[rgb]{0.56,0.35,0.01}{\textbf{\textit{#1}}}}
\newcommand{\ConstantTok}[1]{\textcolor[rgb]{0.56,0.35,0.01}{#1}}
\newcommand{\ControlFlowTok}[1]{\textcolor[rgb]{0.13,0.29,0.53}{\textbf{#1}}}
\newcommand{\DataTypeTok}[1]{\textcolor[rgb]{0.13,0.29,0.53}{#1}}
\newcommand{\DecValTok}[1]{\textcolor[rgb]{0.00,0.00,0.81}{#1}}
\newcommand{\DocumentationTok}[1]{\textcolor[rgb]{0.56,0.35,0.01}{\textbf{\textit{#1}}}}
\newcommand{\ErrorTok}[1]{\textcolor[rgb]{0.64,0.00,0.00}{\textbf{#1}}}
\newcommand{\ExtensionTok}[1]{#1}
\newcommand{\FloatTok}[1]{\textcolor[rgb]{0.00,0.00,0.81}{#1}}
\newcommand{\FunctionTok}[1]{\textcolor[rgb]{0.13,0.29,0.53}{\textbf{#1}}}
\newcommand{\ImportTok}[1]{#1}
\newcommand{\InformationTok}[1]{\textcolor[rgb]{0.56,0.35,0.01}{\textbf{\textit{#1}}}}
\newcommand{\KeywordTok}[1]{\textcolor[rgb]{0.13,0.29,0.53}{\textbf{#1}}}
\newcommand{\NormalTok}[1]{#1}
\newcommand{\OperatorTok}[1]{\textcolor[rgb]{0.81,0.36,0.00}{\textbf{#1}}}
\newcommand{\OtherTok}[1]{\textcolor[rgb]{0.56,0.35,0.01}{#1}}
\newcommand{\PreprocessorTok}[1]{\textcolor[rgb]{0.56,0.35,0.01}{\textit{#1}}}
\newcommand{\RegionMarkerTok}[1]{#1}
\newcommand{\SpecialCharTok}[1]{\textcolor[rgb]{0.81,0.36,0.00}{\textbf{#1}}}
\newcommand{\SpecialStringTok}[1]{\textcolor[rgb]{0.31,0.60,0.02}{#1}}
\newcommand{\StringTok}[1]{\textcolor[rgb]{0.31,0.60,0.02}{#1}}
\newcommand{\VariableTok}[1]{\textcolor[rgb]{0.00,0.00,0.00}{#1}}
\newcommand{\VerbatimStringTok}[1]{\textcolor[rgb]{0.31,0.60,0.02}{#1}}
\newcommand{\WarningTok}[1]{\textcolor[rgb]{0.56,0.35,0.01}{\textbf{\textit{#1}}}}
\usepackage{longtable,booktabs,array}
\newcounter{none} % for unnumbered tables
\usepackage{calc} % for calculating minipage widths
% Correct order of tables after \paragraph or \subparagraph
\usepackage{etoolbox}
\makeatletter
\patchcmd\longtable{\par}{\if@noskipsec\mbox{}\fi\par}{}{}
\makeatother
% Allow footnotes in longtable head/foot
\IfFileExists{footnotehyper.sty}{\usepackage{footnotehyper}}{\usepackage{footnote}}
\makesavenoteenv{longtable}
\usepackage{graphicx}
\makeatletter
\newsavebox\pandoc@box
\newcommand*\pandocbounded[1]{% scales image to fit in text height/width
  \sbox\pandoc@box{#1}%
  \Gscale@div\@tempa{\textheight}{\dimexpr\ht\pandoc@box+\dp\pandoc@box\relax}%
  \Gscale@div\@tempb{\linewidth}{\wd\pandoc@box}%
  \ifdim\@tempb\p@<\@tempa\p@\let\@tempa\@tempb\fi% select the smaller of both
  \ifdim\@tempa\p@<\p@\scalebox{\@tempa}{\usebox\pandoc@box}%
  \else\usebox{\pandoc@box}%
  \fi%
}
% Set default figure placement to htbp
\def\fps@figure{htbp}
\makeatother
\setlength{\emergencystretch}{3em} % prevent overfull lines
\providecommand{\tightlist}{%
  \setlength{\itemsep}{0pt}\setlength{\parskip}{0pt}}
\usepackage{bookmark}
\IfFileExists{xurl.sty}{\usepackage{xurl}}{} % add URL line breaks if available
\urlstyle{same}
\hypersetup{
  pdftitle={Analysis of a first long training},
  pdfauthor={Daniel Stornetta (dss2q)},
  hidelinks,
  pdfcreator={LaTeX via pandoc}}

\title{Analysis of a first long training}
\author{Daniel Stornetta (dss2q)}
\date{}

\begin{document}
\maketitle

{
\setcounter{tocdepth}{2}
\tableofcontents
}
\newpage

\section{Libraries}\label{libraries}

\begin{Shaded}
\begin{Highlighting}[]
\FunctionTok{library}\NormalTok{(tidyverse)}
\end{Highlighting}
\end{Shaded}

\begin{verbatim}
## -- Attaching core tidyverse packages ------------------------ tidyverse 2.0.0 --
## v dplyr     1.1.4     v readr     2.1.5
## v forcats   1.0.0     v stringr   1.5.1
## v ggplot2   3.5.2     v tibble    3.2.1
## v lubridate 1.9.4     v tidyr     1.3.1
## v purrr     1.0.4     
## -- Conflicts ------------------------------------------ tidyverse_conflicts() --
## x dplyr::filter() masks stats::filter()
## x dplyr::lag()    masks stats::lag()
## i Use the conflicted package (<http://conflicted.r-lib.org/>) to force all conflicts to become errors
\end{verbatim}

\section{Loading data}\label{loading-data}

Load in data from an overnight run of the agent on 2025-11-24.

This is from
\url{https://github.com/GITHUB-USER-0/SMB_RL/tree/50c9b0ed68e16ba5045b846bdbb408ce94178a41}.

Hyperparameters:

\begin{itemize}
\tightlist
\item
  FRAME\_WIDTH = 256
\item
  VTRIM = 36
\item
  HTRIM = 36
\item
  HTRIM\_RIGHT = 16
\item
  TRIM\_FRAME\_HEIGHT = 204
\item
  TRIM\_FRAME\_WIDTH = 204
\item
  ADJ\_FRAME\_HEIGHT = 100
\item
  ADJ\_FRAME\_WIDTH = 100
\item
  BUFFER\_SIZE = 1000
\item
  SEED = None
\item
  ROM = v0
\item
  BATCH\_SIZE = 32
\item
  GAMMA = 0.99
\item
  LEARNING\_RATE = 0.0001
\item
  ACTION\_SPACE\_IN\_USE = {[}{[}`right'{]}, {[}`NOOP'{]}, {[}`right',
  `B'{]}, {[}`right', `A'{]}, {[}`A'{]}, {[}`down'{]}{]}
\end{itemize}

\begin{Shaded}
\begin{Highlighting}[]
\NormalTok{dat }\OtherTok{=} \FunctionTok{read\_csv}\NormalTok{(}\StringTok{"./perEpisodeRewards.csv"}\NormalTok{)}
\end{Highlighting}
\end{Shaded}

\begin{verbatim}
## Rows: 12798 Columns: 4
## -- Column specification --------------------------------------------------------
## Delimiter: ","
## chr (1): course
## dbl (2): episode, cumulativeReward
## lgl (1): flag_get
## 
## i Use `spec()` to retrieve the full column specification for this data.
## i Specify the column types or set `show_col_types = FALSE` to quiet this message.
\end{verbatim}

Courses were randomly selected across the entire game.

\begin{itemize}
\tightlist
\item
  episode : training episode
\item
  cumulativeReward : the reward at the end of the episode using the
  default reward function
\item
  course : the world-stage pairing, eg., ``1-1'' corresponds to world 1,
  stage 1, the first course of the game. (For additional details see
  \texttt{/documentation/on\ different\ SMB\ courses.ipynb})
\end{itemize}

\begin{Shaded}
\begin{Highlighting}[]
\NormalTok{dat }\SpecialCharTok{\%\textgreater{}\%} \FunctionTok{head}\NormalTok{()}
\end{Highlighting}
\end{Shaded}

\begin{verbatim}
## # A tibble: 6 x 4
##   episode cumulativeReward course flag_get
##     <dbl>            <dbl> <chr>  <lgl>   
## 1       0              324 6-3    FALSE   
## 2       1              981 2-3    FALSE   
## 3       2             1085 8-1    FALSE   
## 4       3              186 8-3    FALSE   
## 5       4              192 7-1    FALSE   
## 6       5              173 1-4    FALSE
\end{verbatim}

A categorization of levels accessed from the
\href{https://www.mariowiki.com/Super_Mario_Bros.}{Mario Bros.~Wiki}
available under
\href{https://creativecommons.org/licenses/by-sa/3.0/}{CC BY-SA 3.0
license}.

I have added a subcategorization of ``Puzzle'' for three courses based
on their inclusion of looping segments of the course that can repeat
until the player runs out of time.

\begin{Shaded}
\begin{Highlighting}[]
\NormalTok{courseMap }\OtherTok{=} \FunctionTok{read\_csv}\NormalTok{(}\StringTok{"level\_categorization.csv"}\NormalTok{) }\SpecialCharTok{\%\textgreater{}\%}
  \FunctionTok{mutate}\NormalTok{(}\AttributeTok{courseType =} \FunctionTok{if\_else}\NormalTok{(course }\SpecialCharTok{\%in\%} \FunctionTok{c}\NormalTok{(}\StringTok{"4{-}4"}\NormalTok{, }\StringTok{"7{-}4"}\NormalTok{, }\StringTok{"8{-}4"}\NormalTok{), }\FunctionTok{paste0}\NormalTok{(courseType, }\StringTok{", Puzzle"}\NormalTok{), courseType))}
\end{Highlighting}
\end{Shaded}

\begin{verbatim}
## Rows: 32 Columns: 2
## -- Column specification --------------------------------------------------------
## Delimiter: ","
## chr (2): course, courseType
## 
## i Use `spec()` to retrieve the full column specification for this data.
## i Specify the column types or set `show_col_types = FALSE` to quiet this message.
\end{verbatim}

\begin{Shaded}
\begin{Highlighting}[]
\NormalTok{courseMap}
\end{Highlighting}
\end{Shaded}

\begin{verbatim}
## # A tibble: 32 x 2
##    course courseType 
##    <chr>  <chr>      
##  1 1-1    Ground     
##  2 1-2    Underground
##  3 1-3    Athletic   
##  4 1-4    Castle     
##  5 2-1    Ground     
##  6 2-2    Underwater 
##  7 2-3    Athletic   
##  8 2-4    Castle     
##  9 3-1    Ground     
## 10 3-2    Ground     
## # i 22 more rows
\end{verbatim}

\newpage

\section{Overview of training
outcomes}\label{overview-of-training-outcomes}

Comments:

\begin{itemize}
\tightlist
\item
  Outside of sparse high-reward events, it appears that overall rewards
  were relatively low and stable.
\item
  It appears that there was just one level that managed to reach the
  flag pole (terminal state for a level).
\item
  Occasional episodes had very high levels of reward. (\emph{This will
  ultimately align with the hypothesis I had generated about puzzle
  levels, see `documentation/on different SMB courses.ipynb'}).
\end{itemize}

\begin{Shaded}
\begin{Highlighting}[]
\NormalTok{episodeCount }\OtherTok{=} \FunctionTok{nrow}\NormalTok{(dat)}

\NormalTok{dat.success }\OtherTok{=}\NormalTok{ dat }\SpecialCharTok{\%\textgreater{}\%} \FunctionTok{filter}\NormalTok{(flag\_get }\SpecialCharTok{==} \ConstantTok{TRUE}\NormalTok{)}
\NormalTok{dat.failure }\OtherTok{=}\NormalTok{ dat }\SpecialCharTok{\%\textgreater{}\%} \FunctionTok{filter}\NormalTok{(flag\_get }\SpecialCharTok{==} \ConstantTok{FALSE}\NormalTok{)}

\FunctionTok{ggplot}\NormalTok{() }\SpecialCharTok{+}
  \FunctionTok{geom\_point}\NormalTok{(}\AttributeTok{data =}\NormalTok{ dat.failure, }\FunctionTok{aes}\NormalTok{(}\AttributeTok{x =}\NormalTok{ episode, }\AttributeTok{y =}\NormalTok{ cumulativeReward)) }\SpecialCharTok{+}
  \FunctionTok{geom\_point}\NormalTok{(}\AttributeTok{data =}\NormalTok{ dat.success, }\FunctionTok{aes}\NormalTok{(}\AttributeTok{x =}\NormalTok{ episode, }\AttributeTok{y =}\NormalTok{ cumulativeReward), }\AttributeTok{color =} \StringTok{"green"}\NormalTok{, }\AttributeTok{size =} \FloatTok{2.5}\NormalTok{) }\SpecialCharTok{+}
  \FunctionTok{labs}\NormalTok{(}\AttributeTok{x =} \StringTok{"Episode"}\NormalTok{,}
       \AttributeTok{y =} \StringTok{"Cumulative Reward (a.u.)"}\NormalTok{,}
       \AttributeTok{title =} \StringTok{"Per{-}episode rewards in Super Mario Bros (1985)"}\NormalTok{,}
       \AttributeTok{caption =} \FunctionTok{paste0}\NormalTok{(}\StringTok{"Enlarged green observation reflects the only successfully completed level."}\NormalTok{,}
                        \StringTok{"}\SpecialCharTok{\textbackslash{}n}\StringTok{N="}\NormalTok{, episodeCount))}
\end{Highlighting}
\end{Shaded}

\begin{center}\includegraphics{smb_one_long_training_files/figure-latex/unnamed-chunk-5-1} \end{center}
\newpage

The view of the results when ignoring some of the larger outliers
suggests no significant increase in performance over time.

\begin{Shaded}
\begin{Highlighting}[]
\NormalTok{dat }\SpecialCharTok{\%\textgreater{}\%}
  \FunctionTok{filter}\NormalTok{(cumulativeReward }\SpecialCharTok{\textless{}} \DecValTok{2500}\NormalTok{) }\SpecialCharTok{\%\textgreater{}\%}
  \FunctionTok{mutate}\NormalTok{(}\AttributeTok{episode\_1000 =}  \FunctionTok{floor}\NormalTok{(episode }\SpecialCharTok{/} \DecValTok{1000}\NormalTok{)) }\SpecialCharTok{\%\textgreater{}\%}
  \FunctionTok{mutate}\NormalTok{(}\AttributeTok{episode\_1000 =} \FunctionTok{factor}\NormalTok{(episode\_1000)) }\SpecialCharTok{\%\textgreater{}\%}
  \FunctionTok{ggplot}\NormalTok{(}\FunctionTok{aes}\NormalTok{(}\AttributeTok{x =}\NormalTok{ episode\_1000, }\AttributeTok{y =}\NormalTok{ cumulativeReward)) }\SpecialCharTok{+}
  \FunctionTok{geom\_boxplot}\NormalTok{() }\SpecialCharTok{+}
  \FunctionTok{labs}\NormalTok{(}\AttributeTok{x =} \StringTok{"Episode (1,000s grouped)"}\NormalTok{,}
       \AttributeTok{y =} \StringTok{"Cumulative Reward (a.u.)"}\NormalTok{,}
       \AttributeTok{title =} \StringTok{"Rewards per 1,000 episodes in Super Mario Bros (1985)"}\NormalTok{,}
       \AttributeTok{subtitle =} \StringTok{"Rewards greater than 2,500 not inclusive"}\NormalTok{,}
       \AttributeTok{caption =} \FunctionTok{paste0}\NormalTok{(}\StringTok{"Enlarged green observation reflects the only successfully completed level."}\NormalTok{,}
                        \StringTok{"}\SpecialCharTok{\textbackslash{}n}\StringTok{N="}\NormalTok{, episodeCount))}
\end{Highlighting}
\end{Shaded}

\begin{center}\includegraphics{smb_one_long_training_files/figure-latex/unnamed-chunk-6-1} \end{center}

\newpage

\section{Per-course behavior}\label{per-course-behavior}

A look at per-course behavior provides insight to some of the largest
outliers and provides evidence to support a hypothesis generated before
testing.

Namely, the reward structure could allow for degenerate reward signaling
on `puzzle levels' that allow for infinitely scrolling courses when the
agent takes a looping path, see
\texttt{/documentation/on\ different\ SMB\ courses.ipynb}.

\begin{Shaded}
\begin{Highlighting}[]
\NormalTok{dat.nonPuzzle }\OtherTok{=}\NormalTok{ dat }\SpecialCharTok{\%\textgreater{}\%} \FunctionTok{filter}\NormalTok{(}\SpecialCharTok{!}\NormalTok{course }\SpecialCharTok{\%in\%} \FunctionTok{c}\NormalTok{(}\StringTok{"4{-}4"}\NormalTok{, }\StringTok{"7{-}4"}\NormalTok{, }\StringTok{"8{-}4"}\NormalTok{))}
\NormalTok{dat.puzzle }\OtherTok{=}\NormalTok{ dat }\SpecialCharTok{\%\textgreater{}\%} \FunctionTok{filter}\NormalTok{(course }\SpecialCharTok{\%in\%} \FunctionTok{c}\NormalTok{(}\StringTok{"4{-}4"}\NormalTok{, }\StringTok{"7{-}4"}\NormalTok{, }\StringTok{"8{-}4"}\NormalTok{))}

\FunctionTok{ggplot}\NormalTok{() }\SpecialCharTok{+}
  \FunctionTok{geom\_boxplot}\NormalTok{(}\AttributeTok{data =}\NormalTok{ dat.nonPuzzle, }\FunctionTok{aes}\NormalTok{(}\AttributeTok{x =}\NormalTok{ course, }\AttributeTok{y =}\NormalTok{ cumulativeReward)) }\SpecialCharTok{+}
  \FunctionTok{geom\_boxplot}\NormalTok{(}\AttributeTok{data =}\NormalTok{ dat.puzzle, }\FunctionTok{aes}\NormalTok{(}\AttributeTok{x =}\NormalTok{ course, }\AttributeTok{y =}\NormalTok{ cumulativeReward), }\AttributeTok{color =} \StringTok{"orange"}\NormalTok{) }\SpecialCharTok{+}
  \FunctionTok{labs}\NormalTok{(}\AttributeTok{x =} \StringTok{"Episode"}\NormalTok{,}
       \AttributeTok{y =} \StringTok{"Cumulative Reward (a.u.)"}\NormalTok{,}
       \AttributeTok{title =} \StringTok{"Rewards per course in Super Mario Bros (1985)"}\NormalTok{,}
       \AttributeTok{subtitle =} \StringTok{""}\NormalTok{,}
       \AttributeTok{caption =} \FunctionTok{paste0}\NormalTok{(}\StringTok{"Episodes with puzzle designs highlighted in orange"}\NormalTok{)) }\SpecialCharTok{+}
  \FunctionTok{theme}\NormalTok{(}\AttributeTok{axis.text.x =} \FunctionTok{element\_text}\NormalTok{(}\AttributeTok{angle =} \DecValTok{90}\NormalTok{))}
\end{Highlighting}
\end{Shaded}

\begin{center}\includegraphics{smb_one_long_training_files/figure-latex/unnamed-chunk-7-1} \end{center}
\newpage

\begin{Shaded}
\begin{Highlighting}[]
\NormalTok{dat }\SpecialCharTok{\%\textgreater{}\%}
  \FunctionTok{left\_join}\NormalTok{(courseMap) }\SpecialCharTok{\%\textgreater{}\%}
  \FunctionTok{ggplot}\NormalTok{(}\FunctionTok{aes}\NormalTok{(}\AttributeTok{x =}\NormalTok{ courseType, }\AttributeTok{y =}\NormalTok{ cumulativeReward, }\AttributeTok{color =}\NormalTok{ courseType)) }\SpecialCharTok{+}
  \FunctionTok{geom\_boxplot}\NormalTok{() }\SpecialCharTok{+}
  \CommentTok{\#facet\_wrap( \textasciitilde{} courseType, ncol = 2) +}
  \CommentTok{\#coord\_cartesian(ylim = c(0, 2500)) +}
  \FunctionTok{theme}\NormalTok{(}\AttributeTok{axis.text.x =} \FunctionTok{element\_text}\NormalTok{(}\AttributeTok{angle =} \DecValTok{90}\NormalTok{)) }\SpecialCharTok{+}
  \FunctionTok{labs}\NormalTok{(}\AttributeTok{x =} \StringTok{"Course type"}\NormalTok{,}
       \AttributeTok{y =} \StringTok{"Cumulative Reward (a.u.)"}\NormalTok{,}
       \AttributeTok{title =} \StringTok{"Rewards by course type"}\NormalTok{,}
       \AttributeTok{caption =} \FunctionTok{paste0}\NormalTok{(}\StringTok{"NB., puzzle subcategory added by me, other categorizations as per the Mario Bros Wiki,"}\NormalTok{,}
       \StringTok{"CC BY{-}SA 3.0.}\SpecialCharTok{\textbackslash{}n}\StringTok{https://www.mariowiki.com/Super\_Mario\_Bros."}\NormalTok{))}
\end{Highlighting}
\end{Shaded}

\begin{verbatim}
## Joining with `by = join_by(course)`
\end{verbatim}

\begin{center}\includegraphics{smb_one_long_training_files/figure-latex/unnamed-chunk-8-1} \end{center}
\newpage

\section{Periodic values from saved
episodes.}\label{periodic-values-from-saved-episodes.}

\begin{Shaded}
\begin{Highlighting}[]
\NormalTok{dat }\SpecialCharTok{\%\textgreater{}\%}
  \FunctionTok{filter}\NormalTok{(episode }\SpecialCharTok{\%\%} \DecValTok{1000} \SpecialCharTok{==} \DecValTok{0}\NormalTok{) }\SpecialCharTok{\%\textgreater{}\%}
\NormalTok{  knitr}\SpecialCharTok{::}\FunctionTok{kable}\NormalTok{()}
\end{Highlighting}
\end{Shaded}

{\def\LTcaptype{none} % do not increment counter
\begin{longtable}[]{@{}rrll@{}}
\toprule\noalign{}
episode & cumulativeReward & course & flag\_get \\
\midrule\noalign{}
\endhead
\bottomrule\noalign{}
\endlastfoot
0 & 324 & 6-3 & FALSE \\
1000 & 154 & 3-2 & FALSE \\
2000 & 136 & 3-2 & FALSE \\
3000 & 166 & 1-4 & FALSE \\
4000 & 891 & 8-3 & FALSE \\
5000 & 646 & 2-1 & FALSE \\
6000 & 1219 & 4-1 & FALSE \\
7000 & 949 & 8-3 & FALSE \\
8000 & 227 & 3-3 & FALSE \\
9000 & 5496 & 4-4 & FALSE \\
10000 & 440 & 7-2 & FALSE \\
11000 & 373 & 4-1 & FALSE \\
12000 & 333 & 2-3 & FALSE \\
\end{longtable}
}

\newpage

\section{Summary table}\label{summary-table}

Summary table by course.

\begin{Shaded}
\begin{Highlighting}[]
\NormalTok{summaryTable }\OtherTok{=}\NormalTok{ dat }\SpecialCharTok{\%\textgreater{}\%}
  \FunctionTok{group\_by}\NormalTok{(course) }\SpecialCharTok{\%\textgreater{}\%}
  \FunctionTok{summarize}\NormalTok{(}\AttributeTok{mean\_reward =} \FunctionTok{mean}\NormalTok{(cumulativeReward),}
            \AttributeTok{sd\_reward =} \FunctionTok{sd}\NormalTok{(cumulativeReward),}
            \AttributeTok{count =} \FunctionTok{n}\NormalTok{()) }\SpecialCharTok{\%\textgreater{}\%}
  \FunctionTok{mutate}\NormalTok{(}\AttributeTok{mean\_reward =} \FunctionTok{round}\NormalTok{(mean\_reward, }\DecValTok{2}\NormalTok{),}
         \AttributeTok{sd\_reward =} \FunctionTok{round}\NormalTok{(sd\_reward, }\DecValTok{2}\NormalTok{))}

\NormalTok{summaryTable }\SpecialCharTok{\%\textgreater{}\%}
\NormalTok{  knitr}\SpecialCharTok{::}\FunctionTok{kable}\NormalTok{()}
\end{Highlighting}
\end{Shaded}

{\def\LTcaptype{none} % do not increment counter
\begin{longtable}[]{@{}lrrr@{}}
\toprule\noalign{}
course & mean\_reward & sd\_reward & count \\
\midrule\noalign{}
\endhead
\bottomrule\noalign{}
\endlastfoot
1-1 & 472.10 & 328.58 & 379 \\
1-2 & 310.91 & 253.63 & 381 \\
1-3 & 275.80 & 98.17 & 423 \\
1-4 & 246.30 & 132.84 & 393 \\
2-1 & 372.98 & 201.99 & 390 \\
2-2 & 734.25 & 507.59 & 409 \\
2-3 & 603.28 & 358.75 & 401 \\
2-4 & 270.85 & 99.03 & 416 \\
3-1 & 477.08 & 195.37 & 398 \\
3-2 & 307.11 & 317.31 & 409 \\
3-3 & 266.63 & 92.94 & 389 \\
3-4 & 252.88 & 93.35 & 417 \\
4-1 & 489.06 & 331.04 & 418 \\
4-2 & 181.59 & 104.46 & 401 \\
4-3 & 319.61 & 54.55 & 380 \\
4-4 & 153.84 & 537.59 & 420 \\
5-1 & 218.03 & 166.08 & 408 \\
5-2 & 411.60 & 214.51 & 386 \\
5-3 & 253.58 & 101.45 & 388 \\
5-4 & 264.73 & 75.75 & 392 \\
6-1 & 364.28 & 207.22 & 411 \\
6-2 & 455.09 & 233.26 & 403 \\
6-3 & 267.12 & 81.00 & 393 \\
6-4 & 225.27 & 93.38 & 427 \\
7-1 & 391.57 & 232.65 & 422 \\
7-2 & 564.34 & 381.81 & 366 \\
7-3 & 476.61 & 269.86 & 385 \\
7-4 & 465.36 & 1658.35 & 418 \\
8-1 & 291.72 & 273.46 & 384 \\
8-2 & 262.15 & 209.74 & 410 \\
8-3 & 691.15 & 330.24 & 400 \\
8-4 & 116.33 & 156.68 & 381 \\
\end{longtable}
}

\newpage
\twocolumn

\section{Commentary}\label{commentary}

Based on a manual review of episodes traces (ie., reviewing saved image
frames played together in a movie.)

A manual review of episode 0 suggests that it rapidly is valuing the
rightward movement.

A manual review of episode 9,000 provides several observations:

\begin{itemize}
\item
  This is a puzzle level, and the agent is indeed proceeding rightward
  on a wrong path, allowing it to continue to accumulate rewards.
\item
  The reward function uses the delta in x\_position to calculate
  x\_velocity. However, in stages that loop, there are sudden shifts in
  x\_position, jumping from approximately 1,060 (highest observed 1,064)
  to a thousand less. Thus, this results in single reward steps of
  1,000. This could be argued to be a bug of the environment, but also a
  suggestion that there should be greater reward
  stabilization/normalization to prevent learning too much from these
  blips. More simply, the removal of these levels in training may be
  appropriate.
\item
  \begin{itemize}
  \item
    This is supported by how the source code looks at raw RAM values in
    the emulator.Source code:
    \url{https://github.com/sajmon83/gymnasium-super-mario-bros/blob/b8f89fbc2da0d495d87b2d6d1e2d56444df73a4b/gym_super_mario_bros/smb_env.py\#L139}

\begin{verbatim}
def _x_position(self):
"""Return the current horizontal position."""
# add the current page 0x6d to the current x
return int(self.ram[0x6d]) * 0x100 + 
       int(self.ram[0x86])
\end{verbatim}
  \end{itemize}
\item
  There is an additional element of state to the agent, where the
  previously selected actions limit the behavior of a future action. If
  the agent is `holding down' the `A' button to jump either by itself,
  or in combination with others, there is 1) a cooldown, and 2) a
  requirement that the button be released for it to be pressed again.
  This suggests that the combination of prior actions into a model could
  be valuable, not just a stacking of prior states.

  \begin{itemize}
  \tightlist
  \item
    Not that we are seeking to resolve this with dynamic programming,
    but this information suggests that the Markov assumption is violated
    \textbf{if} one is considering only the static frame as the state.
    If momentum values or prior input presses are part of the state then
    one could restore this property.
  \item
    Regardless of other potential fixes, this also suggests the
    importance of annealing the epsilon value over time.
  \end{itemize}
\item
  The agent is predominantly holding down {[}`right', `A'{]}, which
  moves it right and jumps. However, by holding down the `A' button, it
  is unable to jump unless it releases that button for one or more
  frames first. The use of an epsilon value of 0.01 appears to be
  responsible for the jumping observed in this particular episode. Thus,
  the successful dodging that we observe, may simply be a result of
  sampling the system many thousands of times and observing successful
  instances.
\end{itemize}

\subsection{Additional comments on
7-4}\label{additional-comments-on-7-4}

It is interesting to note that there are no instances of negative reward
in the course 7-4. This, in spite of hazards being placed near the start
of the level. One possible interpretation of this is that the value of
the remaining time is counterbalancing the lack of x\_position progress,
as it seems implausible that the agent is not in fact dying in some
instances. This suggests that for post-mortem analysis, one should have
the x\_position and y\_position values stored as well, as they are
helpful (notwithstanding the problems of looping x\_position levels,
which arguably might benefit from being removed.)

It begs the question of whether a reward signal based on the x\_position
of the agent, relative to the end of the level might be more
informative, albeit less generic. Moreover, that would require leaking
information about the level from the model to the agent, something that
it should not know from the environment.

\onecolumn

\begin{Shaded}
\begin{Highlighting}[]
\NormalTok{dat }\SpecialCharTok{\%\textgreater{}\%}
  \FunctionTok{filter}\NormalTok{(course }\SpecialCharTok{==} \StringTok{"7{-}4"}\NormalTok{) }\SpecialCharTok{\%\textgreater{}\%}
  \FunctionTok{filter}\NormalTok{(cumulativeReward }\SpecialCharTok{\textless{}} \DecValTok{0}\NormalTok{)}
\end{Highlighting}
\end{Shaded}

\begin{verbatim}
## # A tibble: 0 x 4
## # i 4 variables: episode <dbl>, cumulativeReward <dbl>, course <chr>,
## #   flag_get <lgl>
\end{verbatim}

\begin{Shaded}
\begin{Highlighting}[]
\NormalTok{dat }\SpecialCharTok{\%\textgreater{}\%}
  \FunctionTok{filter}\NormalTok{(course }\SpecialCharTok{==} \StringTok{"7{-}4"}\NormalTok{) }\SpecialCharTok{\%\textgreater{}\%}
  \FunctionTok{filter}\NormalTok{(cumulativeReward }\SpecialCharTok{\textless{}} \DecValTok{5000}\NormalTok{) }\SpecialCharTok{\%\textgreater{}\%}
  \FunctionTok{ggplot}\NormalTok{(}\FunctionTok{aes}\NormalTok{(}\AttributeTok{x =}\NormalTok{ episode, }\AttributeTok{y =}\NormalTok{ cumulativeReward)) }\SpecialCharTok{+}
  \FunctionTok{geom\_point}\NormalTok{() }\SpecialCharTok{+}
  \FunctionTok{labs}\NormalTok{(}\AttributeTok{x =} \StringTok{"Episode"}\NormalTok{,}
       \AttributeTok{y =} \StringTok{"Cumulative Reward (a.u.)"}\NormalTok{,}
       \AttributeTok{title =} \StringTok{"Rewards in course 7{-}4"}\NormalTok{,}
       \AttributeTok{subtitle =} \StringTok{"Outliers/degenerate rewards removed"}\NormalTok{) }\SpecialCharTok{+}
  \FunctionTok{coord\_cartesian}\NormalTok{(}\AttributeTok{ylim =} \FunctionTok{c}\NormalTok{(}\SpecialCharTok{{-}}\DecValTok{100}\NormalTok{, }\DecValTok{500}\NormalTok{))}
\end{Highlighting}
\end{Shaded}

\begin{center}\includegraphics{smb_one_long_training_files/figure-latex/unnamed-chunk-14-1} \end{center}

\begin{Shaded}
\begin{Highlighting}[]
\NormalTok{dat }\SpecialCharTok{\%\textgreater{}\%}
  \FunctionTok{filter}\NormalTok{(course }\SpecialCharTok{==} \StringTok{"7{-}4"}\NormalTok{) }\SpecialCharTok{\%\textgreater{}\%}
  \FunctionTok{ggplot}\NormalTok{(}\FunctionTok{aes}\NormalTok{(}\AttributeTok{x =}\NormalTok{ episode, }\AttributeTok{y =}\NormalTok{ cumulativeReward)) }\SpecialCharTok{+}
  \FunctionTok{geom\_point}\NormalTok{() }\SpecialCharTok{+}
  \FunctionTok{labs}\NormalTok{(}\AttributeTok{x =} \StringTok{"Episode"}\NormalTok{,}
       \AttributeTok{y =} \StringTok{"Cumulative Reward (a.u.)"}\NormalTok{,}
       \AttributeTok{title =} \StringTok{"Rewards in course 7{-}4"}\NormalTok{,}
       \AttributeTok{subtitle =} \StringTok{""}\NormalTok{)  }
\end{Highlighting}
\end{Shaded}

\begin{center}\includegraphics{smb_one_long_training_files/figure-latex/unnamed-chunk-14-2} \end{center}

\end{document}
